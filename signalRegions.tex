\section{Control and signal regions}
 
In order to define a set of sensitive signal regions in terms of the search variables \mtll, \mtbb and \mtlblb we first exclude the
region $\mtll < 100$ \GeV which will serve as a control region and normalization region for the \ttbar background estimation.
Furthermore, the same-flavor and opposite-flavor search regions will have a different background composition because backgrounds where the two leptons
stem from a Z boson will not contribute to the opposite flavor channel. We therefore split in same-flavor (SF) and opposite-flavor (OF) regions.
 
In order to optimize the sensitivity of the analysis while ensuring reasonable statistical uncertainties in simulated samples, we defined a set of signal regions in the following way.
The region $\mtll > 100$ \GeV is sliced in bins of the other two variables starting with a relatively fine binning of 50 GeV and a large number of signal regions.
In this setting, we estimate the signal and background yields purely from simulation and produce a limit plot including all uncertainties except those related to the data-driven background estimation procedures (discussed later). Signal regions with a vanishing background estimation were excluded in this procedure as those would otherwise have biased the result towards finer binning.
Then, we collapse signal regions starting at the highest bins in each of the variables until there are no more empty regions. At each step, we re-evaluate the simulation-based contour in order
to balance small losses of sensitivity with the ensuing reduction in the number of signal regions. The process showed that a 100 \GeV binning is sufficient except for the first threshold in \mtll which should be at around 140 \GeV. We furthermore checked that splitting
the same-flavor channels into $\mu$ and e is not beneficial. The relatively small number of resulting signal regions in \mtll, \mtbb and \mtlblb is listed in Table~\ref{regions80X}. It reflects an excellent compromise between sensitivity of the analysis and feasibility of the background estimation.
 
 \begin{table}
  \center
  \begin{tabular}{c|c|c|c}
               & \mtll                          & \mtlblb                     & \mtbb \\
    \hline
    0          &                                & $0 \leq \mtlblb \leq 100$   & $70 \leq \mtbb \leq 170$ \\
    1          &                                & $0 \leq \mtlblb \leq 100$   & $170 \leq \mtbb$ \\
    2          & $100 \leq \mtll \leq 140$      & $100 \leq \mtlblb \leq 200$ & $70 \leq \mtbb \leq 170$ \\
    3          &                                & $100 \leq \mtlblb \leq 200$ & $170 \leq \mtbb$ \\
    4          &                                & $200 \leq \mtlblb$          & $70 \leq \mtbb \leq 170$ \\
    5          &                                & $200 \leq \mtlblb$          & $170 \leq \mtbb$ \\
    \hline
    6          &                                & $0 \leq \mtlblb \leq 100$   & $70 \leq \mtbb \leq 170$ \\
    7          &                                & $0 \leq \mtlblb \leq 100$   & $170 \leq \mtbb$ \\
    8          & $140 \leq \mtll \leq 240$      & $100 \leq \mtlblb \leq 200$ & $70 \leq \mtbb \leq 170$ \\
    9          &                                & $100 \leq \mtlblb \leq 200$ & $170 \leq \mtbb$ \\
    10         &                                & $200 \leq \mtlblb$          & $70 \leq \mtbb \leq 170$ \\
    11         &                                & $200 \leq \mtlblb$          & $170 \leq \mtbb$ \\
    \hline
    12         & $240 \leq \mtll$               & $0 \leq \mtlblb$ & $70 \leq \mtbb$ \\
  \end{tabular}
  \caption{Division of signal regions in bins of \mtll, \mtbb and \mtlblb}
  \label{regions80X}
\end{table}

