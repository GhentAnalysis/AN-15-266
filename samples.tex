\section{Samples}

  \subsection{Data samples}

    This analysis uses 2016 pp collision data at $\sqrt{s}$ = 13 TeV, corresponding to an integrated luminosity of 12.9\fbinv. Events are selected from following double-lepton and single-lepton
    primary data sets:

    \begin{itemize}
      \item DoubleEG: two electron channel ($ee$)
      \item DoubleMuon: two muon channel ($\mu\mu$)
      \item MuonEG: one electron and one muon channel ($e\mu$)
      \item SingleElectron: backup for $ee$ and $e\mu$ channels
      \item SingleMuon: backup for $\mu\mu$ and $e\mu$ channels
    \end{itemize}

    For each of the primary data sets, runs 2016B, 2016C and 2016D are considered:
    \begin{itemize}
      \item {\verb /<PD>/Run2016B-PromptReco-v2/MINIAOD } (run 272007-275376)
      \item {\verb /<PD>/Run2016C-PromptReco-v2/MINIAOD } (run 275657-276283)
      \item {\verb /<PD>/Run2016D-PromptReco-v2/MINIAOD } (run 276315-276811)
     %\item {\verb /<PD>/Run2016E-PromptReco-v2/MINIAOD } (run 276831-277420)
     %\item {\verb /<PD>/Run2016F-PromptReco-v2/MINIAOD } (run 277772-)
    \end{itemize}
    and processed using the {\verb Cert_271036-276811_13TeV_PromptReco_Collisions16_JSON } golden JSON file where all CMS subdetectors are flagged as good.


  \subsection{Simulation}
    Background samples have been produced in the Spring16 miniAODv2 campaign and have an average pile-up of $\mu = 20$ and a bunch crossing spacing of 25ns.
    The backgrounds are listed in Table \ref{samples-backgrounds2016} along with their theoretical cross sections.
    \begin{table}
  \center
  \tiny
  \begin{tabular}{l|l|l}
     process                & dataset path & $\sigma \cdot BR$ (pb)\\
     \hline
     Drell-Yan              & \verb /DYJetsToLL_M-5to50_TuneCUETP8M1_13TeV-madgraphMLM-pythia8/[Spring16mAOD]-v1/MINIAODSIM \; (for $H_T < 100$ GeV)          & 71310 \\ 
                            & \verb /DYJetsToLL_M5to50_HT-100to200_TuneCUETP8M1_13TeV-madgraphMLM-pythia8/[Spring16mAOD]-v1/MINIAODSIM                        & 224.2 \\
                            & \verb /DYJetsToLL_M5to50_HT-100to200_TuneCUETP8M1_13TeV-madgraphMLM-pythia8/[Spring16mAOD]_etx1-v1/MINIAODSIM                   & 224.2 \\
                            & \verb /DYJetsToLL_M5to50_HT-200to400_TuneCUETP8M1_13TeV-madgraphMLM-pythia8/[Spring16mAOD]-v1/MINIAODSIM                        & 37.2 \\
                            & \verb /DYJetsToLL_M5to50_HT-200to400_TuneCUETP8M1_13TeV-madgraphMLM-pythia8/[Spring16mAOD]_ext1-v1/MINIAODSIM                   & 37.2 \\
                            & \verb /DYJetsToLL_M5to50_HT-400to600_TuneCUETP8M1_13TeV-madgraphMLM-pythia8/[Spring16mAOD]-v1/MINIAODSIM                        & 3.581 \\
                            & \verb /DYJetsToLL_M5to50_HT-600toInf_TuneCUETP8M1_13TeV-madgraphMLM-pythia8/[Spring16mAOD]-v1/MINIAODSIM                        & 1.124 \\
                            & \verb /DYJetsToLL_M5to50_HT-600toInf_TuneCUETP8M1_13TeV-madgraphMLM-pythia8/[Spring16mAOD]_ext1-v1/MINIAODSIM                   & 1.124 \\
                            & \verb /DYJetsToLL_M-50_TuneCUETP8M1_13TeV-madgraphMLM-pythia8/[Spring16mAOD]_ext1-v1/MINIAODSIM \; (for $H_T < 100$ GeV)        & 5765.4 \\
                            & \verb /DYJetsToLL_M-50_HT-100to200_TuneCUETP8M1_13TeV-madgraphMLM-pythia8/[Spring16mAOD]-v1/MINIAODSIM                          & 171.5 \\
                            & \verb /DYJetsToLL_M-50_HT-100to200_TuneCUETP8M1_13TeV-madgraphMLM-pythia8/[Spring16mAOD]_ext1-v1/MINIAODSIM                     & 171.5 \\
                            & \verb /DYJetsToLL_M-50_HT-200to400_TuneCUETP8M1_13TeV-madgraphMLM-pythia8/[Spring16mAOD]-v1/MINIAODSIM                          & 52.58 \\
                            & \verb /DYJetsToLL_M-50_HT-200to400_TuneCUETP8M1_13TeV-madgraphMLM-pythia8/[Spring16mAOD]_ext1-v1/MINIAODSIM                     & 52.58 \\
                            & \verb /DYJetsToLL_M-50_HT-400to600_TuneCUETP8M1_13TeV-madgraphMLM-pythia8/[Spring16mAOD]-v1/MINIAODSIM                          & 6.761 \\
                            & \verb /DYJetsToLL_M-50_HT-400to600_TuneCUETP8M1_13TeV-madgraphMLM-pythia8/[Spring16mAOD]_ext1-v1/MINIAODSIM                     & 6.761 \\
                            & \verb /DYJetsToLL_M-50_HT-600toInf_TuneCUETP8M1_13TeV-madgraphMLM-pythia8/[Spring16mAOD]-v1/MINIAODSIM                          & 2.72 \\
                            & \verb /DYJetsToLL_M-50_HT-600toInf_TuneCUETP8M1_13TeV-madgraphMLM-pythia8/[Spring16mAOD]_ext1-v1/MINIAODSIM                     & 2.72 \\
     $t\bar{t}$       %      & \verb /TTTo2L2Nu_13TeV-powheg/[Spring16mAOD]-v1/MINIAODSIM                                                                      & 87.31 \\
                            & \verb /TTTo2L2Nu_13TeV-powheg/[Spring16mAOD*]_ext3-v1/MINIAODSIM						                      & 87.31 \\
     single $t$             & \verb /ST_tW_top_5f_inclusiveDecays_13TeV-powheg-pythia8_TuneCUETP8M1/[Spring16mAOD]-v2/MINIAODSIM			      & 35.6 \\
                            & \verb /ST_tW_antitop_5f_inclusiveDecays_13TeV-powheg-pythia8_TuneCUETP8M1/[Spring16mAOD]-v1/MINIAODSIM                          & 35.6 \\
                      %      & \verb /ST_t-channel_4f_leptonDecays_13TeV-amcatnlo-pythia8_TuneCUETP8M1/[Spring16mAOD]-v1/MINIAODSIM                            & 70.31 \\
                      %      & \verb /ST_t-channel_4f_leptonDecays_13TeV-amcatnlo-pythia8_TuneCUETP8M1/[Spring16mAOD]_ext1-v1/MINIAODSIM                       & 70.31 \\
     $t\bar{t}Z$      %      & \verb /TTZToQQ_TuneCUETP8M1_13TeV-amcatnlo-pythia8/[Spring16mAOD]-v1/MINIAODSIM                                                 & 0.5297 \\
                      %      & \verb /TTZToLLNuNu_M-10_TuneCUETP8M1_13TeV-amcatnlo-pythia8/[Spring16mAOD]-v1/MINIAODSIM                                        & 0.2529 \\
                            & \verb /ttZJets_13TeV_madgraphMLM/[Spring16mAOD]]-v1/MINIAODSIM                                                                  & 0.7826 \\
     $t\bar{t}W$            & \verb /TTWJetsToLNu_TuneCUETP8M1_13TeV-amcatnloFXFX-madspin-pythia8/[Spring16mAOD]-v1/MINIAODSIM                                & 0.2043 \\
                            & \verb /TTWJetsToQQ_TuneCUETP8M1_13TeV-amcatnloFXFX-madspin-pythia8/[Spring16mAOD]-v1/MINIAODSIM                                 & 0.4062 \\
     $t\bar{t}H$            & \verb /ttHJetToNonbb_M125_13TeV_amcatnloFXFX_madspin_pythia8_mWCutfix/[Spring16mAOD*]_ext1-v2\MINIAODSIM                        & 0.2151 \\
     $tZq$                  & \verb /tZq_ll_4f_13TeV-amcatnlo-pythia8_TuneCUETP8M1/[Spring16mAOD]-v1/MINIAODSIM                                               & 0.0758 \\
                     %       & \verb /tZq_nunu_4f_13TeV-amcatnlo-pythia8_TuneCUETP8M1/[Spring16mAOD]-v1/MINIAODSIM 					    & 0.1379 \\
%    $t\bar{t}\gamma$       & \verb /TTGJets_TuneCUETP8M1_13TeV-amcatnloFXFX-madspin-pythia8/[Spring16mAOD]-v1/MINIAODSIM                                     & 3.697 \\
     $WW, WZ, ZZ$           & \verb /WWTo1L1Nu2Q_13TeV_amcatnloFXFX_madspin_pythia8/[Spring16mAOD]-v1/MINIAODSIM                                              & 49.997 \\
                            & \verb /WWToLNuQQ_13TeV-powheg/[Spring16mAOD]-v1/MINIAODSIM                                                                      & 43.53 \\
                            & \verb /WWToLNuQQ_13TeV-powheg/[Spring16mAOD]_ext1-v1/MINIAODSIM                                                                 & 43.53 \\
                            & \verb /WZTo1L3Nu_13TeV_amcatnloFXFX_madspin_pythia8/[Spring16mAOD]-v1/MINIAODSIM                                                & 3.054 \\
                            & \verb /WZTo1L1Nu2Q_13TeV_amcatnloFXFX_madspin_pythia8/[Spring16mAOD]-v1/MINIAODSIM                                              & 10.71 \\
                            & \verb /WZTo2L2Q_13TeV_amcatnloFXFX_madspin_pythia8/[Spring16mAOD]-v1/MINIAODSIM                                                 & 5.60 \\
                            & \verb /WZTo3LNu_TuneCUETP8M1_13TeV-powheg-pythia8/[Spring16mAOD]-v1/MINIAODSIM                                                  & 4.42965 \\
                            & \verb /ZZTo2L2Q_13TeV_amcatnloFXFX_madspin_pythia8/[Spring16mAOD]-v1/MINIAODSIM                                                 & 3.28 \\
                            & \verb /ZZTo2Q2Nu_13TeV_amcatnloFXFX_madspin_pythia8/[Spring16mAOD]-v1/MINIAODSIM                                                & 4.04 \\
                            & \verb /VVTo2L2Nu_13TeV_amcatnloFXFX_madspin_pythia8/[Spring16mAOD]-v1/MINIAODSIM                                                & 11.95 \\
     tribosons              & \verb /WWW_4F_TuneCUETP8M1_13TeV-amcatnlo-pythia8/[Spring16mAOD]-v1/MINIAODSIM                                                  & 0.2086 \\
                            & \verb /WWZ_TuneCUETP8M1_13TeV-amcatnlo-pythia8/[Spring16mAOD]-v1/MINIAODSIM                                                     & 0.1651 \\
                            & \verb /WZZ_TuneCUETP8M1_13TeV-amcatnlo-pythia8/[Spring16mAOD]-v1/MINIAODSIM                                                     & 0.05565 \\
                            & \verb /ZZZ_TuneCUETP8M1_13TeV-amcatnlo-pythia8/[Spring16mAOD]-v1/MINIAODSIM                                                     & 0.01398 \\
  \end{tabular}\\
  \vspace{2mm}
    \verb [Spring16mAOD] = \verb RunIISpring16MiniAODv2-PUSpring16_80X_mcRun2_asymptotic_2016_miniAODv2_v0 \\
    \verb [Spring16mAOD*] = \verb RunIISpring16MiniAODv2-PUSpring16RAWAODSIM_80X_mcRun2_asymptotic_2016_miniAODv2_v0 \\
  \caption{Background samples}
  \label{samples-backgrounds2016}
\end{table}

    Signal samples are listed in Table \ref{samples-signals2016}.
    \begin{table}
  \center
  \tiny
  \begin{tabular}{l|l}
     process                & dataset \\ 
     \hline
     T2tt signal            & \verb /SMS-T2tt_mStop-150to250_TuneCUETP8M1_13TeV-madgraphMLM-pythia8/[Spring16mAODFS]-v1/MINIAODSIM \\
                            & \verb /SMS-T2tt_mStop-250to350_TuneCUETP8M1_13TeV-madgraphMLM-pythia8/[Spring16mAODFS]-v1/MINIAODSIM \\ 
                            & \verb /SMS-T2tt_mStop-350to400_TuneCUETP8M1_13TeV-madgraphMLM-pythia8/[Spring16mAODFS]-v1/MINIAODSIM \\ 
                            & \verb /SMS-T2tt_mStop-400to1200_TuneCUETP8M1_13TeV-madgraphMLM-pythia8/[Spring16mAODFS]-v1/MINIAODSIM \\
                            & \verb /SMS-T2tt_mStop-425_mLSP-325_TuneCUETP8M1_13TeV-madgraphMLM-pythia8/[Spring16mAODFS]-v1/MINIAODSIM \\
                            & \verb /SMS-T2tt_mStop-500_mLSP-325_TuneCUETP8M1_13TeV-madgraphMLM-pythia8/[Spring16mAODFS]-v2/MINIAODSIM \\
                            & \verb /SMS-T2tt_mStop-850_mLSP-100_TuneCUETP8M1_13TeV-madgraphMLM-pythia8/[Spring16mAODFS]-v1/MINIAODSIM \\
  \end{tabular}\\
  \vspace{2mm}
     \verb [Spring16mAODFS] = \verb RunIISpring16MiniAODv2-PUSpring16Fast_80X_mcRun2_asymptotic_2016_miniAODv2_v0 \\
  \caption{FastSim Signal samples scanning $m_{\sTop}$ between 150 and 1200 GeV.}
  \label{samples-signals2016}
\end{table}

